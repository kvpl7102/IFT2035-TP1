\documentclass{article}

\usepackage[french]{babel}
\usepackage[utf8]{inputenc}
\usepackage[T1]{fontenc}
\usepackage{amsfonts}            %For \leadsto
\usepackage{amsmath}             %For \text
\usepackage{fancybox}            %For \ovalbox

\title{Travail pratique \#1}
\author{IFT-2035}

\begin{document}

\maketitle

{\centering \ovalbox{\large ¡¡ Dû le 3 juin à 23h59 !!} \\}

\newcommand \mML {\ensuremath\mu\textsl{ML}}
\newcommand \kw [1] {\textsf{#1}}
\newcommand \id [1] {\textsl{#1}}
\newcommand \punc [1] {\kw{`#1'}}
\newcommand \str [1] {\texttt{"#1"}}
\newenvironment{outitemize}{
  \begin{itemize}
  \let \origitem \item \def \item {\origitem[]\hspace{-18pt}}
}{
  \end{itemize}
}
\newcommand \MAlign [2][t] {
  \begin{array}[#1]{@{}l}
    #2
  \end{array}}

\section{Survol}

Ce TP vise à améliorer la compréhension des langages fonctionnels en
utilisant un langage de programmation fonctionnel (Haskell) et en écrivant
une partie d'un interpréteur d'un langage de programmation fonctionnel (en
l'occurrence une sorte de Lisp).  Les étapes de ce travail sont les suivantes:
\begin{enumerate}
\item Parfaire sa connaissance de Haskell.
\item Lire et comprendre cette donnée.  Cela prendra probablement une partie
  importante du temps total.
\item Lire, trouver, et comprendre les parties importantes du code fourni.
\item Compléter le code fourni.
\item Écrire un rapport.  Il doit décrire \textbf{votre} expérience pendant
  les points précédents: problèmes rencontrés, surprises, choix que vous
  avez dû faire, options que vous avez sciemment rejetées, etc...  Le
  rapport ne doit pas excéder 5 pages.
\end{enumerate}

Ce travail est à faire en groupes de 2 étudiants.  Le rapport, au format
\LaTeX\ exclusivement (compilable sur \texttt{ens.iro}), et le code sont
à remettre par remise électronique avant la date indiquée.  Aucun retard ne
sera accepté.  Indiquez clairement votre nom au début de chaque fichier.

Si un étudiant préfère travailler seul, libre à lui, mais l'évaluation de
son travail n'en tiendra pas compte. %% Si un étudiant ne trouve pas de
%% partenaire, même après en avoir cherché dans le forum de \emph{discussions},
%% il doit me contacter au plus tard le 9 octobre \textbf{avant} le cours.
Des groupes de 3 ou plus sont \textbf{exclus}.

\newpage
\section{Psil: Une sorte de Lisp}

\begin{figure}
  \begin{displaymath}
    \begin{array}{r@{~}r@{~}l@{~~}l}
      \tau &::=& \kw{Int} & \text{Type des nombres entiers} \\
      &\mid& (\tau_1~...~\tau_n \to \tau) & \text{Type d'une fonction}
       \medskip \\
      e &::=& n & \text{Un entier signé en décimal} \\
      &\mid& x & \text{Référence à une variable}  \\
      &\mid& (e_0~e_1~...~e_n) &
           \text{Un appel de fonction (\emph{curried})} \\
      &\mid& (\kw{fun}~x~e) & \text{Une fonction d'un argument} \\
      &\mid& (\kw{let}~((x~e_1))~e_2) &
           \text{Ajout d'une variable locale} \\
      &\mid& ({\kw{:}}~e~\tau) & \text{Annotation de type} \\
      &\mid& + \mid - \mid * \mid / \mid \kw{if0} & 
           \text{Opérations arithmétiques}
       \medskip \\
      d &::=& (\kw{dec}~x~\tau) & \text{Déclaration de variable} \\
      &\mid& (\kw{def}~x~e) & \text{Définition de variable}
      \medskip \\
      p &::=& d_1~...~d_n & \text{Programme} \\
      
%%           \medskip \\
%%       d &::=& (x~e) & %% \mid ((x~x_1~...~x_n)~e)
%%           \text{Déclaration de variable} \\ %% et de fonction
%%       &\mid& (x~\tau~e) & \text{Déclaration de variable avec son type} \\
%%       &\mid& (x~(x_1~\tau_1)~...~(x_n~\tau_n)~\tau~e) &
%%            \text{Déclaration de fonction avec son type}
    \end{array}
  \end{displaymath}
  \caption{Syntaxe de Psil}
  \label{fig:syntaxe}
\end{figure}

Vous allez travailler sur l'implantation d'un langage fonctionnel dont la
syntaxe est inspirée du langage Lisp.  La syntaxe de ce langage est décrite
à la Figure~\ref{fig:syntaxe}.

À remarquer qu'avec la syntaxe de style Lisp, les parenthèses sont
généralement significatives.  Mais contrairement à la tradition de Lisp, le
type des fonctions $(\tau_1 \to \tau_2)$ utilise une syntaxe infixe plutôt
que préfixe.

La forme \kw{let} est utilisée pour donner un nom à une définition
locale.  Elle n'autorise pas la récursion.  Exemple:
\begin{displaymath}
  \MAlign[c]{(\kw{let}~(\MAlign{(x~2))} \\\;\; (+~x~3))} \;\;\;\;\leadsto^*\;\;\;\;5
\end{displaymath}
Un programme Psil est une suite de déclarations.  Les déclarations ont
généralement la forme $(\kw{def}~x~e)$ qui défini une nouvelle variable
globale $x$, mais une telle définition peut aussi être précédée comme en
Haskell d'une déclaration du type de $x$ avec $(\kw{dec}~x~\tau)$.
Cette déclaration est indispensable si $x$ est définie récursivement (i.e.~a
besoin de faire référence à elle-même):
\begin{displaymath}
  \MAlign{
    (\kw{dec}~\id{infloop}~(\kw{Int}~\to~\kw{Int})) \\
    (\kw{def}~\id{infloop}~(\kw{fun}~x~(\id{infloop}~(+~x~1))))
  }
\end{displaymath}


\subsection{Sucre syntaxique}
\label{sec:sucre}

Les fonctions n'ont qu'un seul argument: la syntaxe offre la possibilité de
passer plusieurs arguments, mais ces arguments sont passés par
\emph{currying}.  Plus précisément, les équivalences suivantes sont vraies
pour les expressions:
\begin{displaymath}
  \begin{array}{r@{\;\;\;\;\Longleftrightarrow\;\;\;\;}l}
    (e_0~e_1~e_2~...~e_n) & (..((e_0~e_1)~e_2)...~e_n) \\
%%     (\kw{fn}~(x_1~...~x_n)~e) &
%%         (\kw{fn}~(x_1)~...~(\kw{fn}~(x_n)~e)..) \\
    (\tau_1~...~\tau_n \to \tau) & (\tau_1\to ... (\tau_n\to\tau)..)
  \end{array}
\end{displaymath}
%% De plus, la syntaxe d'une déclaration de fonction est elle aussi du sucre
%% syntaxique, et elle est régie par l'équivalence suivante pour les déclarations:
%% \begin{displaymath}
%%   \begin{array}{r@{\;\;\;\;\Longleftrightarrow\;\;\;\;}l}
%%     ((x~(x_1~\tau_1)~...~(x_n~\tau_n))~\tau~e) &
%%         (x~(\tau_1~...~\tau_n\to\tau)~(\kw{fn}~(x_1~...~x_n)~e))
%%   \end{array}
%% \end{displaymath}

Vous allez utilisez ces équivalences dans la première partie du code qui va
convertir le code d'entrée du format générique \kw{Sexp} au format
``Lambda'', un format beaucoup plus pratique pour implanter la suite,
i.e. la vérification des types et l'évaluation.

\subsection{Sémantique statique}

\newcommand \CheckKind [1] {\vdash #1}
\newcommand \CheckType [2][\Gamma] {#1 \vdash #2 \Leftarrow }
\newcommand \InferType [2][\Gamma] {#1 \vdash #2 \Rightarrow }
\newcommand \BothType [2][\Gamma] {#1 \vdash #2 ~:~}
\newcommand \Axiom [2] { \mbox{}\hspace{10pt}\nolinebreak\ensuremath{\displaystyle
    \frac{#1}{#2}}\nolinebreak\hspace{10pt}\mbox{} \medskip}
\newcommand \HA {\hspace{15pt}}
\renewcommand \: {\!:\!}

\begin{figure}
  \begin{minipage}{\columnwidth}
    \noindent \centering

    \Axiom{}{\InferType{n}{\kw{Int}}}
    \Axiom{\Gamma(x) = \tau}{\InferType{x}{\tau}}
    \Axiom{\CheckType{e}{\tau}}
          {\InferType{(\kw{:}~e~\tau)}{\tau}}
    \Axiom{\InferType{e}{\tau}}
          {\CheckType{e}{\tau}}
    \Axiom{\InferType{e_1}{(\tau_1\to\tau_2)}
           \HA
           \CheckType{e_2}{\tau_1}}
          {\InferType{(e_1~e_2)}{\tau_2}}
    \Axiom{\CheckType[\Gamma,x\:\tau_1]{e}{\tau_2}}
          {\CheckType{(\kw{fun}~x~e)}{(\tau_1\to\tau_2)}}
    \Axiom{\InferType{e_1}{\tau_1} \HA \InferType[\Gamma,x\:\tau_1]{e_2}{\tau_2}}
          {\InferType{(\kw{let}~((x~e_1))~e_2)}{\tau_2}}
    \mbox{}
  \end{minipage}
  \caption{Règles de typage}
  \label{fig:typing}
\end{figure}

Une des différences les plus notoires entre Lisp et Psil est que Psil est
typé statiquement.  Les règles de typage (voir Figure~\ref{fig:typing})
utilisent deux jugements $\CheckType{e}{\tau}$ et $\InferType{e}{\tau}$ qui
disent que l'expression $e$ est typée correctement et a type $\tau$.  Dans ces
règles, $\Gamma$ représente le contexte de typage, c'est à dire qu'il contient le
type de toutes les variables auxquelles $e$ peut faire référence.

Ce genre de typage est dit ``bidirectionnel'': $\CheckType{e}{\tau}$ est
utilisé lorsqu'on connaît déjà le type attendu de $e$ et qu'on veut donc
vérifier que $e$ a effectivement ce type, et on utilise $\InferType{e}{\tau}$
lorsqu'au contraire on ne connaît pas à priori le type de $e$ et donc la
règle est sensée ``synthétiser'' le type $\tau$ en analysant $e$.

Cette approche permet de réduire un peu la quantité d'annotations de types
pour le programmeur Psil, en profitant de manière ``opportuniste'' de
l'information de typage disponible, tout en étant beaucoup plus simple
à implanter qu'une inférence de types comme celle de Haskell.

Il manque les règles de typage des opérations arithmétiques, car elles sont
traitées simplement comme des ``variables prédéfinies'' qui
sont donc incluse dans le contexte $\Gamma$ initial.

La deuxième partie du travail est d'implanter la vérification de types, donc
de transformer ces règles en un morceau de code Haskell.  Un détail
important pour cela est que le but fondamental de la vérification de types
n'est pas de trouver le type d'une expression mais plutôt de trouver
d'éventuelles erreurs de typage, donc il est important de \emph{tout} vérifier.

\subsection{Sémantique dynamique}

Les valeurs manipulées à l'exécution par notre langage sont seulement les
entiers et les fonctions.

Les règles de réductions primitives ($\beta$) sont les suivantes:
\begin{displaymath}
  \begin{array}{r@{\;\;\;\leadsto\;\;\;}l}
    ((\kw{fun}~x~e)~v) & e[v/x] \\
    (\kw{let}~((x~v))~e) & e[v/x]
  \end{array}
\end{displaymath}
où la notation $e[v/x]$ représente l'expression $e$ dans un
environnement où la variable $x$ prend la valeur $v$.  L'usage de $v$ dans
la règle ci-dessus indique qu'il s'agit bien d'une valeur plutôt que d'une
expression non encore évaluée.
%% 
En plus de la règle ci-dessus, les différentes primitives se
comportent comme suit:
\begin{displaymath}
  \begin{array}{r@{\;\;\;\leadsto\;\;\;}l}
    (+~n_1~n_2) & n_1 + n_2 \\
    (-~n_1~n_2) & n_1 - n_2 \\
    (*~n_1~n_2) & n_1 \times n_2 \\
    (/~n_1~n_2) & n_1 \div n_2 \\
    (\kw{if0}~n_1~n_2~n_3) & \left\{
    \begin{array}{ll}
      n_2 & \text{si $n_1 = 0$} \\
      n_3 & \text{sinon}
    \end{array}\right.
    \\
  \end{array}
\end{displaymath}
%%
Donc il s'agit d'une variante du $\lambda$-calcul, sans grande surprise.
La portée est statique et l'ordre d'évaluation est présumé être ``par
valeur'', mais vu que le langage est pur, la différence n'est pas très
importante pour ce travail.

\section{Implantation}

L'implantation du langage fonctionne en plusieurs phases:
\begin{enumerate}
\item Une première phase d'analyse lexicale et syntaxique transforme le code
  source en une représentation décrite ci-dessous, appelée \id{Sexp} (pour
  ``syntactic expression'') dans le code.  C'est une sorte de
  proto-ASA (arbre de syntaxe abstraite).  Cette partie vous est offerte.
\item Une deuxieme phase, appelée élaboraion, termine l'analyse syntaxique et
  commence la compilation, en transformant cet arbre en un vrai arbre de
  syntaxe abstraite dans la représentation appellée ``Lambda'' (ou \id{Lexp}
  pour ``lambda expression'') dans le code.  Comme mentionné, cette phase
  commence déjà la compilation vu que le langage \id{Lexp} n'est pas
  identique à notre langage source.  En plus de terminer l'analyse
  syntaxique, cette phase élimine le sucre syntaxique (i.e.~applique les
  règles de la forme $...\Longleftrightarrow...$).
\item Une troisième phase, appelée \id{check/synth}, vérifie que le code est
  correctement typé et en trouve son type.
\item Finalement, une fonction \id{eval} procède à l'évaluation des
  expressions par interprétation, et \id{process\_decl} l'utilise pour
  évaluer toutes les déclarations du programme.
\end{enumerate}

Votre travail consiste à compléter ces phases.

\subsection{Analyse lexicale et syntaxique: \id{Sexp}}

L'analyse lexicale et (une première partie de l'analyse) syntaxique est déjà
implantée pour vous.  Elle est plus permissive %% générale
que nécessaire et accepte n'importe quelle expression de la
forme suivante:
%%
\begin{displaymath}
  \begin{array}{r@{~}r@{~}l}
    e & ::= & n ~\mid~ x 
    ~\mid~ \punc{(}~\{\,e\,\}%% ~[\,.~e\,]
     ~\punc{)}
  \end{array}
\end{displaymath}
%%
\begin{outitemize}
\item $n$ est un entier signé en décimal.  \\
  Il est représenté dans l'arbre en Haskell par: $\kw{Snum}~n$.
\item $x$ est un symbole qui peut être composé d'un nombre quelconque de
  caractères alphanumériques et/ou de ponctuation.  Par exemple $\punc+$ est
  un symbole, $\punc{<=}$ est un symbole, $\punc{voiture}$ est un symbole,
  et $\punc{a+b}$ est aussi un symbole.  Dans l'arbre en Haskell, un symbole
  est représenté par: $\kw{Ssym}~x$.
\item $\punc{(}~\{\,e\,\}%% ~[\,.~e\,]
  ~\punc{)}$ est une liste d'expressions.  Dans l'arbre en Haskell, les
  listes d'expressions sont représentées par des listes simplement chaînées
  constituées de paires $\kw{Scons}~\id{left}~\id{right}$ et du marqueur de
  début $\kw{Snil}$.  \id{right} est le dernier élément de la liste et
  \id{left} est le reste de la liste (i.e.~ce qui le précède).
\end{outitemize}
%%
Par exemple l'analyseur syntaxique transforme l'expression \texttt{(+ 2 3)}
dans l'arbre suivant en Haskell:
\begin{displaymath}
  \kw{Scons}~\MAlign{
    (\kw{Scons}~\MAlign{
      (\kw{Scons}~\MAlign{
        {\kw{Snil}} \\
        (\kw{Ssym}~\str{+}))} \\
      (\kw{Snum}~2))} \\
    (\kw{Snum}~3)}
\end{displaymath}
%%
L'analyseur lexical considère qu'un caractère $\punc{;}$ commence un
commentaire, qui se termine alors à la fin de la ligne.

\subsection{La représentation intermédiaire ``Lambda''}

Cette représentation intermédiaire est le cœur de notre implémentation de
Psil, et représente une sorte d'arbre de syntaxe abstraite.  Dans cette
représentation, \kw{+}, \kw{-}, ... sont simplement des variables
prédéfinies, et le sucre syntaxique n'est plus disponible, donc par exemple
les appels de fonctions ne prennent plus qu'un seul argument.

Elle est définie par les types suivants:

\begin{verbatim}
data Lexp = Lnum Int            -- Constante entière.
          | Lvar Var            -- Référence à une variable.
          | Lhastype Lexp Ltype -- Annotation de type.
          | Lapp Lexp Lexp      -- Appel de fonction, avec un argument.
          | Llet Var Lexp Lexp  -- Déclaration de variable locale.
          | Lfun Var Lexp       -- Fonction anonyme.
          deriving (Show, Eq)
\end{verbatim}

\begin{verbatim}
data Ltype = Lint
           | Larw Ltype Ltype   -- Type "arrow" des fonctions.
           deriving (Show, Eq)

data Ldec = Ldec Var Ltype      -- Déclaration globale.
          | Ldef Var Lexp       -- Définition globale.
          deriving (Show, Eq)
\end{verbatim}

\subsection{L'environnement d'exécution}

Le code fourni défini aussi l'environnement initial d'exécution, qui
contient les fonctions prédéfinies du langage telles que l'addition, la
soustraction, etc.  Il est défini comme une table qui associe à chaque
identificateur prédéfini la valeur (de type \id{Value}) associée.

\section{Cadeaux}

Comme mentionné, l'analyseur lexical et l'analyseur syntaxique sont
déjà fournis.  Dans le fichier \texttt{Psil.hs}, vous trouverez les
déclarations suivantes:
\begin{outitemize}
\item \id{Sexp} est le type des arbres, il défini les différents noeuds qui
  peuvent y apparaître.
\item \id{readSexp} est la fonction d'analyse syntaxique.
\item \id{showSexp} est un pretty-printer qui imprime une expression sous sa
  forme ``originale''.
\item \id{Lexp} est le type de la représentation intermédiaire du même nom.
\item \id{s2l} est la fonction qui ``élabore'' une expression de type
  \id{Sexp} en \id{Lexp}.
\item \id{check} et \id{synth} sont les fonctions qui vérifient et
  synthétisent (infèrent) le type d'une expression.
\item \id{tenv0} est l'environnement de typage initial.
\item \id{venv0} est l'environnement d'évaluation initial.
\item \id{Value} est le type du résultat de l'évaluation d'une expression.
\item \id{eval} est la fonction d'évaluation qui transforme une expression
  de type \id{Lexp} en une valeur de type \id{Value}.
%% \item \id{evalSexp} est une fonction qui combine les phases ci-dessus pour
%%   évaluer une \id{Sexp}.
\item \id{run} est la fonction principale qui lie le tout; elle prend un nom
  de fichier et évalue sur toutes les déclarations trouvées dans ce fichier.
\end{outitemize}

Voilà ci-dessous un exemple de session interactive sur une machine GNU/Linux,
avec le code fourni:
\begin{verbatim}
% ghci psil.hs
GHCi, version 8.8.4: https://www.haskell.org/ghc/  :? for help
[1 of 1] Compiling Main             ( psil.hs, interpreted )

psil.hs:310:1: warning: [-Wincomplete-patterns]
    Pattern match(es) are non-exhaustive
    In an equation for ‘eval’:
        Patterns not matched:
            _ (Lhastype _ _)
            _ (Lapp _ _)
            _ (Llet _ _ _)
            _ (Lfun _ _)
    |
310 | eval _venv (Lnum n) = Vnum n
    | ^^^^^^^^^^^^^^^^^^^^^^^^^^^^...

psil.hs:322:1: warning: [-Wincomplete-patterns]
    Pattern match(es) are non-exhaustive
    In an equation for ‘process_decl’:
        Patterns not matched: ((_, _), Just (_, _), _) (Ldef _ _)
    |
322 | process_decl (env, Nothing, res) (Ldec x t) = (env, Just (x,t), res)
    | ^^^^^^^^^^^^^^^^^^^^^^^^^^^^^^^^^^^^^^^^^^^^^^^^^^^^^^^^^^^^^^^^^^^^...
Ok, one module loaded.
*Main> run "exemples.psil"
  2 : Lint
  <fonction> : Larw Lint (Larw Lint Lint)
  *** Exception: Expression Psil inconnue: (+ 2)
CallStack (from HasCallStack):
  error, called at psil.hs:223:10 in main:Main
*Main> 
\end{verbatim}

Lorsque votre travail sera fini, il ne devrait plus y avoir
d'avertissements, et \texttt{run} devrait renvoyer plus de valeurs que
juste les deux ci-dessus et terminer sans erreurs.

\section{À faire}

Vous allez devoir compléter l'implantation de ce langage, c'est à dire
compléter \id{s2l}, \id{check}, \id{eval}, ...
Je recommande de le faire en avançant dans
\texttt{exemples.psil} plutôt que d'essayer de compléter tout \id{s2l} avant
de commencer à attaquer la suite.
Ceci dit, libre à vous de choisir l'ordre qui vous plaît.

De même je vous recommande fortement de travailler en binôme (\emph{pair
programming}) plutôt que de vous diviser le travail, vu que la difficulté
est plus dans la compréhension que dans la quantité de travail.

Le code contient des indications des endroits que vous devez modifiez.
Généralement cela signifie qu'il ne devrait pas être nécessaire de faire
d'autres modifications, sauf ajouter des définitions auxiliaires.  Vous pouvez
aussi modifier le reste du code, si vous le voulez, mais il faudra alors
justifier ces modifications dans votre rapport en expliquant pourquoi cela
vous a semblé nécessaire.

Vous devez aussi fournir un fichier de tests \texttt{tests.psil}, qui, tout
comme \texttt{exemples.psil} devrait non seulement contenir du code Psil
mais aussi indiquer les valeurs et types qui lui corresponde (à votre avis).
Il doit contenir au moins 5 tests que \emph{vous} avez écrits et qui
devraient tester différents recoins de Psil.

\subsection{Remise}

Pour la remise, vous devez remettre trois fichiers (\texttt{psil.hs},
\texttt{tests.psil}, et \texttt{rapport.tex}) par la page Moodle (aussi
nommé StudiUM) du cours.  Assurez-vous que le rapport compile correctement
sur \texttt{ens.iro} (auquel vous pouvez vous connecter par SSH).

\section{Détails}

\begin{itemize}
\item La note sera divisée comme suit: 25\% pour le rapport, 15\% pour les
  tests, 60\% pour le code Haskell (i.e. élaboration, vérification des
  types, et évaluation).
\item Tout usage de matériel (code ou texte) emprunté à quelqu'un d'autre
  (trouvé sur le web, ...) doit être dûment mentionné, sans quoi cela sera
  considéré comme du plagiat.
\item Le code ne doit en aucun cas dépasser 80 colonnes.
\item Vérifiez la page web du cours, pour d'éventuels errata, et d'autres
  indications supplémentaires.
\item La note est basée d'une part sur des tests automatiques, d'autre part
  sur la lecture du code, ainsi que sur le rapport.  Le critère le plus
  important, est que votre code doit se comporter de manière correcte.
  Ensuite, vient la qualité du code: plus c'est simple, mieux c'est.
  S'il y a beaucoup de commentaires, c'est généralement un symptôme que le
  code n'est pas clair; mais bien sûr, sans commentaires le code (même
  simple) est souvent incompréhensible.  L'efficacité de votre code est sans
  importance, sauf %% pour \id{eval} où vous devez faire attention à faire
  %% autant de travail que possible avant de recevoir \id{venv}, ou bien sûr de
  %% manière plus générale
  si votre code utilise un algorithme vraiment
  particulièrement ridiculement inefficace.
\end{itemize}

\end{document}
